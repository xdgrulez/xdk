\documentclass[halfparskip]{scrartcl}

\usepackage[latin9]{inputenc}
\usepackage[T1]{fontenc}
\usepackage[british]{babel}

\newcommand*{\ID}{%
  \textsc{id}}
\newcommand*{\MESSAGE}[1]{%
  \bigskip\par\fbox{#1}}
\newcommand*{\META}[1]{%
  $\langle$\emph{#1}$\rangle$}
\let\NOTE=\footnote
\newcommand*{\XML}{%
  \textsc{xml}}
\newcommand*{\XMLTAG}[1]{%
  \texttt{<#1 />}}
\newcommand*{\XMLTAGOPEN}[1]{%
  \texttt{<#1>}}
\newcommand*{\XMLTAGCLOSE}[1]{%
  \texttt{</#1>}}

\begin{document}

This document specifies the current draft of the Oracle Protocol.  It
was last modified on 2004-09-28.  Note that the draft is subject to
change.

The Oracle Protocol regulates the communication between the search
engine (the client) and the oracle (the server).  Communication
happens through the sending and receiving of \emph{messages}.  Each
message is a string that can be interpreted as a (possibly nested)
\XML\ element.  Below, boxed messages are messages that the client
sends to the server.  For each message, the potential response
messages (from the server to the client) are listed.

\MESSAGE{\XMLTAG{init}}

This is the message that starts a new communication between the client
and the server.  In later versions of the protocol, it may be used to
negotiate details of the communication, such as the version of the
Oracle Protocol to use, and the domain-specific description language
that the oracle understands.  To implement the Oracle Protocal
according to the current draft, this message should always be
confirmed.

Possible answers:
\begin{itemize}
\item \XMLTAG{confirm} -- The server understands the version of the
  Oracle Protocol and the domain-specific description language that
  the client wants to use.
\item \XMLTAG{reject} -- The client should either re-send the
  \texttt{init} message (using different parameters), or close the
  connection.  [To implement the Oracle Protocol according to the
  current draft, the oracle should not send this message.]
\end{itemize}

\MESSAGE{\XMLTAGOPEN{new id='$i$' parentid='$p$'}\META{description}\XMLTAGCLOSE{new}}

This message tells the oracle that a new search node has been created
as child of the search node with \ID~$p$, and that this new node has
been assigned the \ID~$i$.  By convention, if~$i$ is the root node of
the search tree, the client will make sure that $i=p$.

A \emph{description} is an \XML\ container element of the form
\XMLTAGOPEN{description mode='$t$'}, where~$t$ either is
\texttt{complete} (the description is a complete description of the
current node) or \texttt{diff} (the description is a list of
information that should be added to the description of the parent of
the current node to get a complete description).  It can only be
\texttt{diff} if~$i$ is not the \ID\ of the root of the search tree.
The \XML\ children of the \texttt{description} tag should be
interpreted by the oracle according to the domain-specific
externalisation language negotiated in the \texttt{init} dialogue.

Possible answers:
\begin{itemize}
\item \XMLTAG{confirm} -- The oracle has successfully internalised the
  information about the new search node.
\item \XMLTAG{reject} -- Internalising the new information was either
  not successful or not possible.  The client may choose to resend the
  message, reset the search, or close the connection.
\end{itemize}

\MESSAGE{\XMLTAG{reset}}

This message tells the server to completely forget every information
it inferred from any intermediate \texttt{new} messages, and reset its
internal state such that search can start anew.

Possible answers:
\begin{itemize}
\item \XMLTAG{confirm} -- The server has successfully reset its
  internal state.
\item \XMLTAG{reject} -- Resetting the internal state was either not
  successful or not possible.  The client may choose to resend the
  message, reset the search, or close the connection.
\end{itemize}

\MESSAGE{\XMLTAG{chooseLocal refid='$i$'}}

The client asks for the \ID\ of the search node in which to proceed
search, given that the current node has \ID~$i$.  For example, if~$i$
is a parent node, then the client might expect the oracle to return
the \ID\ of one of $i$'s children as a response to this question.

\begin{itemize}
\item \XMLTAG{next id='$i$'} -- The client should continue search with
  node~$i$.
\item \XMLTAG{noUnexpandedState} -- The server does not know about any
  node at which search could continue.  This can happen, for example.
  if the client asks the \texttt{new} message with a \texttt{refid}
  that denotes a leaf in the search tree.  The client may choose to
  send additional messages, reset the search, or close the connection.
\end{itemize}

\MESSAGE{\XMLTAG{chooseGlobal}}

This message is like the previous one, except that the \ID\ of the
parent node is not taken into account.  This is the message that the
client would usually use if the server implements a global search, in
which the node at which search should be continued does not depend on
the current focus of search be at some particular node.

\begin{itemize}
\item \XMLTAG{next id='$i$'} -- see above
\item \XMLTAG{noUnexpandedState} -- see above
\end{itemize}

\end{document}
