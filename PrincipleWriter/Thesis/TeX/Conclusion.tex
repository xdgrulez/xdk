\chapter{Zusammenfassung und Zuk\"unftige Arbeiten}\label{ch:conc}

\section{Zusammenfassung}

In dieser Arbeit habe ich in den ersten Kapiteln den Extensible
Dependency Grammar Formalismus \cite{DebusmannEtal04SS} erkl\"art, und
wie Grammatiken damit formal entwickelt werden k\"onnen. Grammatiken
bestehen in XDG aus drei Teilen: Einem Multigraphtyp, einem Lexikon
und einer Menge von Prinzipien, die die Wohlgeformtheitsbedingungen
der Multigraph-Modelle der Grammatik regeln. In der neuesten
Formalisierung von XDG \cite{Debusmann07MTS} werden die Prinzipien in
Pr\"adikatenlogik erster Stufe ausgedr\"uckt.  Als Beispiel habe ich
die CSD-Grammatik vorgestellt, die Gebrauch von einem speziell f\"ur
diese Grammatik entwickelten Prinzip macht
(CSD-Prinzip). %Die Erweiterbarkeit der Prinzipienbibliothek ist �lso notwendig.
Als n\"achstes habe ich das XDG Development Kit (XDK)
\cite{DebusmannEtal04MOZ} vorgestellt und beschrieben, wie damit
XDG-Grammatiken implementiert werden k\"onnen. Mit Hilfe der XDK
Description Language, die Bestandteil des XDK ist, k\"onnen der
Multigraphtyp und das Lexikon einer Grammatik analog zur
Formalisierung in XDG aufgeschrieben werden. Die dann noch fehlende
Komponenten einer Grammatik, die Prinzipien, k\"onnen nicht mit der
XDK Description Language beschrieben werden, sondern nur aus einer
Bibliothek von vordefinierten Prinzipien ausgew\"ahlt, oder m\"uhsam
von Hand implementiert werden.

Ich habe gezeigt, wie die im ersten Kapitel vorgestellte CSD-Grammatik
in XDK implementiert wird. Dabei wurde deutlich, wie schwierig es ist,
das XDK um weitere Prinzipien zu erweitern, da diese als Mozart/Oz
Constraints implementiert werden mussten, was nur von Experten der
Mozart/Oz Constraintprogrammierung gemacht werden kann, da auch auf
die Effizienz der Constraints zu achten ist. Das bedeutet, die
Constraints mussten von Hand optimiert werden. Die gro{\ss}e L\"ucke
zwischen der Formalisierung und der Implementierung wurde deutlich.
Die L\"ucke war f\"ur typische Grammatikschreiber, wie z.B.\
Linguisten, die keine Experten in der Mozart/Oz Constraint
Programmierung sind, praktisch un\"uberbr\"uckbar gro{\ss}, so dass
die Erweiterbarkeit des XDK in Punkto Prinzipien praktisch nicht
gegeben war.

Um diese L\"ucke zu schliesen habe ich den PrincipleWriter entwickelt.
Der PrincipleWriter ist ein Werkzeug mit dem Prinzipien analog zur
XDG-Formalisierung als pr\"adikatenlogische Formeln erster Stufe
aufgeschrieben werden k\"onnen. Die Mozart/Oz-Constraints werden dann
vom PrincipleWriter erzeugt, so dass neue Prinzipien leicht
geschrieben werden k\"onnen. Die generierten Mozart/Oz-Constraints
k\"onnen dann einfach ins XDK eingegliedert werden. Damit ist die
Erweiterbarkeit von XDG jetzt auch in der Implementierung, dem XDK,
gegeben.

Um die Prinzipien als logische Formeln aufschreiben zu k\"onnen, habe
ich eine Syntax entwickelt, die PrincipleWriter User Language (PWUL),
mit der Prinzipien analog zur Formalisierung aufgeschrieben werden
k\"onnen. Um das Aufschreiben weiter zu vereinfachen hat der
PrincipleWriter einen Typchecker mit Typinferenz, so dass Typfehler
fr\"uhzeitig aufgedeckt werden k\"onnen und Typannotationen fast nicht
mehr n\"otig sind, was die Prinzipien in PWUL noch dichter an die XDG
Formalisierung bringt.

Wie die Semantik der PrincipleWriter User Language definiert ist, und
wie mit Hilfe der Semantik die in Logik geschriebenen Prinzipien in
Mozart/Oz-Constraints \"ubersetzt werden, habe ich dann gezeigt. Da
die eigentliche Evaluierung die Operatoren der Logik fast direkt in
Mozart/Oz-Constraints \"ubersetzt, wurde deutlich, dass die
generierten Constraints noch optimiert werden m\"ussen, da z.B.\ durch
verschachtelte Quantoren die generierten Prinzipien kaum effizient
genug sind f\"ur den praktischen Einsatz.

Ob es m\"oglich ist, die Prinzipien automatisch so zu optimieren, dass
sie effizient genug f\"ur den Praxiseinsatz sind, war die
Herausforderung. In dem Kapitel \"uber die Optimierung habe ich
gezeigt, wie diese funktioniert, und dass es m\"oglich ist, die
Prinzipien ausreichend gut zu optimieren. Die Optimierung funktioniert
durch Pattern-Matching auf der Mozart/Oz-Darstellung der abstrakten
Syntax und geschieht damit noch vor der Evaluierung. Die Pattern, auf
die gematcht wird, haben versierte Grammatikschreiber bisher auch
schon genutzt, um die Prinzipien zu optimieren. Von deren Erfahrungen
konnte ich profitieren.

Durch den PrincipleWriter habe ich die L\"ucke zwischen der XDG
Formalisierung und der Implementierung (XDK) geschlossen, so dass
jetzt jeder einfach neue Prinzipien entwickeln kann und damit neue
Grammatken schreiben kann, ohne Kenntnisse der Mozart/Oz
Constraintprogrammierung zu haben, und ohne sich Gedanken um die
Optimierung zu machen. Aber auch die Experten profitieren vom
PrincipleWriter, da f\"ur sie das Implementieren von Prinzipien eine
sehr zeitaufw\"andige Arbeit war.

%Einen ersten Praxistest hat der PrincipleWriter auch schon bestanden,
%da Ralph Debusmann mit Hilfe des PrincipleWriters eine Grammatik
%schreiben konnte, deren Implementierung von Hand zu zeitaufw\"andig
%war. Die generierten Prinzipien funktionierten, und waren effizient genug. 

\section{Zuk\"unftige Arbeiten}

Auch wenn die bisher in Praxistests generierten Prinzipien effizient
genug sind f\"ur den Einsatz in der Praxis, bleibt die
Herausforderung, die automatische Optimierung so zu verbessern, dass
die generierten Prinzipien so gut wie die Prinzipien der Mozart/Oz
Experten werden, oder vielleicht sogar noch besser. Dieses Feld bietet
viel Raum f\"ur weitere Forschungen.  Auch die Frage ob es eine
optimale Optimierung gibt, und falls ja, ob diese dann automatisiert
werden kann ist eine interessante Frage.

Bisher ist die Optimierung noch sehr rudiment\"ar durch einfaches
Pattern-Matching realisiert. Die Muster entspringen dem
Erfahrungsschatz von Experten. Um die Optimierung weiter zu
verbessern, k\"onnten weitere Muster gesucht werden und damit das
Pattern-Matching erweitert werden.

Ein Problem beim Optimieren durch reines Pattern-Matching habe ich im
Optimierungskapitel bereits erw\"ahnt. Da verschiedene Formeln
\"aquivalent sein k\"onnen, werden manche Optimierungen nicht
vorgenommen, da kein Muster passt. Beim Optimieren von Hand kann man
solche \"Aquivalenzen erkennen und die Optimierung vornehmen. Um
dieses Problem beim automatisierten Optimieren in den Griff zu
bekommen, m\"usste untersucht werden, ob eine Art Normalform f\"ur die
logischen Formeln gefunden werden kann, in die die Formeln automatisch
\"uberf\"uhrt werden k\"onnen und die \"Aquivalenzen damit aufgel\"ost
werden w\"urden.

Eine andere Idee die Optimierung zu verbessern wird motiviert durch
die Feststellung, dass die urspr\"ungliche Implementierung des
Barierenprinzips exzessiven Gebrauch von Selection-Union-Constraints
macht. Damit ist das Prinzip deutlich schneller, als die automatisch
generierte Variante des PrincipleWriters.

Die eigentliche Idee f\"ur die weitere Optimierung ist folgende:
Praktisch jedes Prinzip verwendet Pfade in irgendeiner Weise. Alle
Pfade starten mit einem Knoten, den wir den Ankerknoten des Prinzips
nennen, und gehen von diesem Knoten weiter. Dadurch sollte es nicht
n\"otig sein \"uber mehr Knoten als \"uber den Ankerknoten zu
quantifizieren. Dies w\"urde bedeuten, dass alle Prinzipien so
implementiert werden k\"onnten, dass ihre Laufzeit linear zur
Knotenmenge w\"are.

Ein kleines Beispiel soll die Idee verdeutlichen. Beim
Climbing-Prinzip:
\begin{verbatim}
forall V::node: forall V1::node:
   dom(V V1 D1) => dom(V V1 D2)
\end{verbatim}
ist der Ankerknoten {\tt V}. Das Prinzip besagt, dass wenn {\tt V1}
von {\tt V} dominiert wird auf der Dimension {\tt D1}, dann muss {\tt
  V1} auch auf der Dimension {\tt D2} von {\tt V} dominiert werden.
Das bedeutet aber nichts anders, als dass die Knoten, die in der
Dimension {\tt D1} von {\tt V} dominiert werden, eine Teilmenge der
Knoten sein m\"ussen, die in der Dimension {\tt D2} von {\tt V}
dominiert werden. Damit kann die zweite Quantifizierung entfernt
werden und Gebrauch des Teilmengen-Constraints gemacht werden:
\begin{verbatim}
forall V::node:
   down_D1(V) subseteq down_D2(V)
\end{verbatim}
Damit hat man eine lineare Implementierung des Prinzips, da nur noch
\"uber einen Knoten quantifiziert wird, w\"ahrend die urspr\"ungliche
Implementierung quadratisch war.

Alle Prinzipien der Prinzipienbibliothek k\"onnten so umgeschrieben
werden, dass sie nur noch \"uber einen Knoten quantifizieren, und
somit linear zur Knotenmenge sind. Ob diese Optimierung automatisiert
werden kann, ist eine interessante Frage und Herausforderung.

%Auch zu kl\"aren w\"are, ob die Quantifzierungen \"uber Kanten
%\"ahnlich eliminiert werden k\"onnen. 



%% I've been thinking about further optimizing the output of PW. The main
%% motivation was the Barriers principle, which in its original version makes
%% heavy use of sets and Select.union to obtain a much more efficient
%% principle as the one I currently get with PW.

%% The basic idea goes like this: every practical principle somehow talks
%% about "paths" in the multigraph. All paths start with one node, which I
%% will call the "anchor node" of the principle, and go on further from it.
%% As a result, it should *not* be necessary to quantify over more than the
%% anchor node, which means that all principles can actually be implemented
%% in a linear fashion!

%% How does this work? Here is a simple example, the climbing principle
%% ("PrincipleWriter/Examples/climbingPW.ul"):

%% forall V::node: forall V1::node:
%%    dom(V V1 D1) =dom(V V1 D2)

%% The anchor is V. It states that if V1 is dominated by V on D1, then it
%% must also be dominated by V on D2. But this means nothing else but that
%% the nodes dominated by V on D1 must be a subset of the nodes dominated by
%% V on D2. That is, we can remove the second node quantification, and make
%% use of the underlying sets:

%% forall V::node:
%%    down_D1(V) subseteq down_D2(V)

%% And voila: we have obtained a linear principle implementation (quantifying
%% only over one node) as opposed to the quadratic one before.

%% I have tried to "play" the optimizer, and have hand-optimized some of the
%% principles in "Solver/Principles/Lib" in this way. Apart from having to
%% quantify over the edge labels (which could be changed as well, I don't
%% know), all these principles are now linear with respect to the nodes.

%% Actually, this optimization is not new - Denys and me have used it
%% intuitively since the beginning. The challenge now is however to equip PW
%% with a means to optimize the output accordingly. So more precisely, the
%% challenge will be to find out how to do this optimization in a
%% compositional fashion.

%% Cheers, Ralph
