\section{Bisherige Arbeiten}

XDG ist ein constraint-basierter Grammatikformalismus, und als solcher
auch vergleichbar mit unifikationsbasierten Formalismen wie Definite
Clause Grammars (DCG) und PATR-II \cite{Shieber84}. Auch hier werden
Wohlgeformtheitsbedingungen durch Constraints beschrieben, allerdings
in einer Programmiersprache (Prolog).

Der Grammatikformalismus Head-driven Phrase Structure Grammar (HPSG)
ist ebenfalls h{\"a}ufig in Prolog implementiert, z.B.\ im
Babel-System \cite{Mueller96}. Neue Wohlgeformtheitsbedingungen werden
hier jedoch, ebenso wie in anderen HPSG-Implementationen wie dem Type
Resolution System (Troll) \cite{GerdemannEtal94}, Attribute Logic
Engine (ALE) \cite{CarpenterPenn94} und TRALE \cite{MeurersEtal02},
durch Unifikation {\"u}ber Merkmalsstrukturen ausgedr{\"u}ckt.

Die Pr{\"a}dikatenlogik erster Stufe, mit der die Constraints von XDG
beschrieben werden, l{\"a}sst sich auch als Modellierungssprache
f{\"u}r Constraintprogrammierung auffassen. Als solche ist sie
vergleichbar mit Sprachen wie die Optimization Programming Language
(OPL) \cite{Hentenryck99}, MiniZinc \cite{NethercoteEtal07}, und
nat{\"u}rlich auch Mozart/Oz \cite{Smolka95}, \cite{Schulte02}.

Das L{\"o}sen von First-Order Formeln mithilfe von
Constraintprogrammierung ist auch in \cite{AptVermeulen02} zu finden.
In \cite{TackEtal06} werden Formeln in Existential Monadic Second
Order Logic (EMSO) in neue Propagatoren {\"u}bersetzt, die auf
endlichen Mengen arbeiten.  Diese Arbeit kommt meiner noch am
n{\"a}chsten. Allerdings setze ich die XDG-Formeln (in
Pr{\"a}dikatenlogik erster Stufe) nicht in neue Propagatoren um,
sondern in Code, der existierende Propagatoren von Mozart/Oz einsetzt.

\section{Gliederung der Arbeit}

Die Arbeit ist wie folgt gegliedert. In Kapitel~\ref{XDG} f{\"u}hre
ich XDG ein, und in Kapitel~\ref{ch:XDK} das XDG Development Kit
(XDK). Diese beiden Kapitel bereiten das Hauptkapitel der Arbeit vor,
Kapitel~\ref{ch:pw}, in dem ich den PrincipleWriter vorstelle.
Kapitel~\ref{ch:conc} schlie{\ss}t die Arbeit ab und gibt einen
Ausblick.
