\begin{titlepage}

\begin{center}
  \begin{tabular}{c}
\LARGE{Universit\"at des Saarlandes}\\
\LARGE{Naturwissenschaftlich-Technische Fakult\"at}\\
\LARGE{Fachrichtung Informatik}\\
\LARGE{Bachelor-Studiengang}
    \\\\\\\\\\
    \\\\\\
\LARGE{\bf{Bachelorarbeit}}\\
\\\\
\huge{\bf{A Principle Compiler}}\\
\huge{\bf{for Extensible Dependency Grammar}}\\\\\\
\LARGE{vorgelegt von}\\\\
\LARGE{\bf{Jochen Setz}}\\\\
\LARGE{am 10.10.2007}\\\\\\\\
\LARGE{angefertigt unter der Leitung von}\\\\
\LARGE{Prof.\ Dr.\ Gert Smolka}\\\\
\LARGE{betreut von}\\\\
\LARGE{Dr.\ Ralph Debusmann}\\\\\\\\\\
\LARGE{begutachtet von}\\\\
\LARGE{Prof.\ Dr.\ Gert Smolka}\\
\LARGE{Dr.\ Ralph Debusmann}
  \end{tabular}
\end{center}



\pagebreak

%%%%%%%%%%%%%%%%%%%%%%%%%%%%%%%%%%%%%%%%%%%%%%%%%%%%%%%%%%%%

\pagestyle{empty}

\begin{flushleft}
  \begin{tabular}{l}
    \\\\\\\\\\\\\\\\
    \\\\\\\\\\\\\\\\
    \\\\\\\\\\\\\\\\
    {\Large{\bf Erkl\"arung}}\\\\
    {\Large{Hiermit erkl\"are ich, dass ich die vorliegende Arbeit
    selbstst\"andig verfasst}} \\
    {\Large und alle verwendeten Quellen angegeben habe.}\\\\
    {\Large{Saarbr\"ucken, den 10.10.2007}}\\\\\\\\\\\\\\\\\\\\\\

    {\Large{\bf Einverst\"andniserkl\"arung}}\\\\
    {\Large{Hiermit erkl\"are ich mich damit einverstanden, dass meine
        Arbeit in den}} \\
    {\Large Bestand der Bibliothek der Fachrichtung Informatik aufgenommen wird.}\\\\
    {\Large{Saarbr\"ucken, den 10.10.2007}}

  \end{tabular}
\end{flushleft}

\pagebreak

%%%%%%%%%%%%%%%%%%%%%%%%%%%%%%%%%%%%%%%%%%%%%%%%%%%%%%%%%%%%

\pagestyle{empty}

\begin{flushleft}
  \begin{tabular}{l}
    \\\\\\\\\\\\\\\\
    \\\\\\\\\\\\\\\\
    \\\\\\\\\\\\\\\\
    \\\\\\\\\\\\\\\\
    \\\\\\\\\\\\\\
    {\Large{\bf Danksagung}}\\\\
    {\Large{Danken m\"ochte ich Dr. Ralph Debumann und Prof. Gert
        Smolka,}}\\
    {\Large{dass sie mir diese Thesis angeboten haben. Vor allem Ralph}}\\ 
    {\Large{geb\"uhrt gro{\ss}er Dank f\"ur die erstklassige Betreuung dieser Arbeit.}}\\\\
    {\Large{Dann m\"ochte ich es nicht verpassen, mich bei meiner Familie und}}\\
    {\Large{meinen Freunden f\"ur ihre Unterst\"utzung zu bedanken.}}\\
\end{tabular}
\end{flushleft}

\pagebreak

\pagestyle{empty}

\begin{center}
  {\sfb\Large{Abstract}}
\end{center}

\vspace{10pt}

Extensible Dependency Grammar (XDG) ist ein neuer,
constraint-basierter Meta-Gram\-matikformalismus
\cite{Debusmann06}. Die Modelle von XDG werden durch Constraints in
Pr{\"a}dikatenlogik erster Stufe charakterisiert, die
\emph{Prinzipien} genannt werden. In der Entwicklungsumgebung f{\"u}r
XDG, dem XDG Development Kit (XDK), sind die Prinzipien als
Constraints auf endlichen Mengen in Mozart/Oz implementiert
\cite{Schulte02}. Allerdings ist die Implementation alles andere als
trivial, und bis jetzt konnte nur eine Minderheit der Benutzer des
XDK, Experten in Mozart/Oz Constraintprogrammierung, dieses auch um
neue Prinzipien erweitern.

In dieser Arbeit entwickle ich ein Programm namens
\emph{PrincipleWriter (PW)}, das Constraints in First-Order Logik
automatisch in Constraints in Mozart/Oz umsetzt. Damit schlie{\ss}e
ich die bisher bestehende L{\"u}cke zwischen der Formalisierung von
XDG und der Implementierung. Jetzt k{\"o}nnen \emph{alle} Benutzer des
XDK dieses um neue Prinzipien erweitern.  Vom PW profitieren aber auch
diejenigen, die die manuelle Umsetzung von Prinzipien in Constraints
beherrschen, da die schnellere und sicherere automatische
{\"U}bersetzung die Grammatikentwicklungszeit deutlich verringert.
All das steigert die Attraktivit{\"a}t des Systems als
Explorationsplattform f{\"u}r dependenzbasierte Grammatikformalismen
erheblich.

\pagebreak
\strut
\thispagestyle{empty}
\newpage

%%%%%%%%%%%%%%%%%%%%%%%%%%%%%%%%%%%%%%%%%%%%%%%%%%%%%%%%%%%%

\pagestyle{empty}
\end{titlepage}
\cleardoublepage
